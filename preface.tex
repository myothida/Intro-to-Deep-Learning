\chapter*{Preface}

This book is based on \textbf{`Introduction to Supervised Machine Learning'} course that I am teaching at the Chiang Mai University.

My interest in AI began in 2003 when I completed a "Literature Review on Face Recognition Methods" as my first undergraduate research project. This project sparked my passion for computer vision and machine learning, leading me to pursue both a Master's and a PhD in the field. Throughout my academic journey, I have published research in journals and a book on "\emph{Contextual Analysis of Videos}".

This book began as a personal project while I was working as an educational consultant in Myanmar five years ago. I would spend four days a week consulting for an international organization, and on my "day off," I taught programming, machine learning, and computer vision to students in my own institute. Initially, I had no plans to write a book as there are already many books available online for free and are constantly updated. I was simply trying to keep my knowledge up-to-date and enjoyed teaching.

However, I noticed that many Myanmar students had limited access to good books due to language barriers, and most of the available books were too advanced for them. This realization led me to write "\emph{Introduction to MATLAB: Learning by Doing}" in the Myanmar language. Over 1,000 copies were sold within a couple of months, and students reported that the book helped them overcome their fear of programming. Encouraged by the positive feedback, I began writing my next book, "\emph{Introduction to Machine Learning in the Myanmar Language}," but due to other commitments, I was unable to complete it after the first chapter.

During the Covid pandemic, I had the opportunity to take a break from my busy reform projects and return to my research in the fields of computer vision and data science. It became clear to me that terms like "AI, Data Science, Machine Learning, and Deep Learning" were being used frequently and that there was an abundance of books, learning materials, and online courses available. However, choosing high-quality resources to learn from is still a challenge for many new learners.

Additionally, I noticed that many young people wanted to learn but did not have internet access. I wanted to make it possible for them to learn about machine learning in a simple way and to be a part of the technology that is changing our daily lives. While teaching in Myanmar, Bhutan and Thailand, I found that students enjoyed my teaching style and asked for me to create YouTube videos and transcripts. All of these requests motivated me to resume work on my pending book, "Introduction to Machine Learning", but this time, I decided to focus solely on "\emph{Supervised Machine Learning Methods}" and to write the book in both Myanmar and English languages, so that it could be useful for my students from Bhutan and Thailand.

The purpose of this book is to document my teachings at Chiang Mai University in a physical form and make it accessible for students with limited resources to learn from. The book aims to provide a comprehensive and easy-to-follow introduction to the fundamental concepts of machine learning methods. It is divided into four parts.

The first chapter provides an overview of the basic questions of machine learning and introduces the Python development environment. The second chapter covers various regression methods and the third chapter discusses different classification methods. The last chapter provides recommendations for continuing the journey of learning machine learning. I believe that hands-on learning is crucial for understanding and thus, the explanations in the book are accompanied by detailed 'Python code' snippets throughout the text. The readers can follow the instructions and run the code on their own computer or an online platform such as Google Colab.

I hope that this book provies a valuable resource for underprivileged youths.

\chapter*{Website}

The drmyothida website contains much additional material, will be available soon at \href{www.drmyothida.org/courses}{www.drmyothida.org/courses/machinelearning}. As students read through \textbf{Introduction to Supervised Machine Learning}, they can go online to take self-grading quizzes and access learning materials such as PowerPoint slides and recorded videos. The complete codes for all the projects are also available at the public
\href{https://github.com/myothida/Intro-To-Supervised-Machine-Learning.git}{\textbf{GitHub Repo}}.
