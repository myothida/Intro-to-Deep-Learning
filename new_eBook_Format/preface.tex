\chapter*{Preface}

%The terms 'AI,' 'deep learning,' and 'machine learning' are not unfamiliar to me, as I am trained as a computer vision and machine learning scientist. My PhD research, which focused on analyzing the contextual information of videos using traditional machine learning techniques, remains relevant, with many readers still citing it as foundational work. However, I took a career hiatus of approximately six years to contribute to the national development of Myanmar. Upon returning to the technological landscape in 2019, I experienced stress and anxiety, fearing I had fallen behind.
%
%Determined to catch up, I quickly familiarized myself with new terminologies and found learning to be more manageable, perhaps due to my increased maturity and experience collaborating with industry professionals. This newfound clarity enabled me to communicate complex concepts more effectively, as evidenced by the success of my previous work, 'Introduction to Supervised Machine Learning (2023),' which sold over 100 copies within a year. Moreover, receiving positive feedback from readers about the accessibility and clarity of my explanations motivated me to embark on writing this teaching book: 'Introduction to Deep Learning.'

My interest in AI began in 2003 when I completed a literature review on face recognition methods as my first undergraduate research project. This project sparked my passion for computer vision and machine learning, leading me to pursue both a Master's and a PhD in the field. Throughout my academic journey, I published research in journals and authored a book titled 'Contextual Analysis of Videos,' all written in English.

Over time, I became aware of the challenges faced by many Myanmar students, who had limited access to quality books due to language barriers and found most available resources too advanced. This realization inspired me to write technology books in Burmese using my own teaching style. My first book, 'Introduction to MATLAB: Learning by Doing,' was published in 2008. The second book, 'Introduction to Supervised Machine Learning,' written in both English and Burmese, was published in 2023. Encouraged by positive feedback on the accessibility and clarity of my explanations, I embarked on writing this teaching book: 'Introduction to Deep Learning.'

This book emerged from one of the most challenging periods of my life, during which the tumultuous situation in Myanmar weighed heavily on my mind, particularly witnessing the struggles of many young people. The purpose of this book is not only to make deep learning accessible to students with limited resources but also to contribute to the support of youths in need. 

This book is structured into four parts to facilitate a comprehensive understanding of deep learning concepts and methodologies. The first chapter serves as a foundational introduction to deep machine learning algorithms. It covers the historical evolution of deep learning, essential mathematical concepts, and prerequisites necessary to comprehend the subsequent chapters.  It also introduces Python libraries commonly used for deep learning, such as TensorFlow and Keras.

Chapter two delves into the architecture and functioning of shallow neural networks. It covers topics such as activation functions, forward and backward propagation, and model optimization techniques. Practical projects demonstrating the application of shallow NNs in regression, binary classification, and multi-class classification tasks are included.  Chapter three explores advanced deep learning architectures, including convolutional neural networks (CNNs), recurrent neural networks (RNNs), and their variants. It discusses the principles behind these architectures and their applications in computer vision, natural language processing, and time-series data analysis.

Chapter four delves into specialized topics and emerging trends in deep learning. It covers areas such as transfer learning, attention mechanisms, AutoML, neural architecture search (NAS), and ethical considerations in deep learning. Through case studies and discussions, this chapter provides insights into advanced methodologies and ethical considerations shaping the future of deep learning.

I believe that hands-on learning is crucial for understanding, and thus, the explanations in the book are accompanied by detailed 'Python code' snippets throughout the text. Readers can follow the instructions and run the code on their own computer or an online platform such as Google Colab.

As you delve into the chapters that follow, I encourage you to approach this journey with an open mind and a spirit of exploration. Whether you are a seasoned expert in the field or a curious novice, there is something to be gained from the insights contained within these pages.

The website: \href{www.drmyothida.org/courses}{www.drmyothida.org/courses/machinelearning}. The complete code for all the projects in this book is also available on the public \href{https://github.com/myothida/Intro-To-Supervised-Machine-Learning.git}{\textbf{GitHub Repo}}.

\chapter*{စာညွှန်း}

စာရေးသူမှာ စင်ကာပူနိုင်ငံ နန်ယန်း တက္ကသိုလ်တွင် အင်ဂျင်နီယာ ဘွဲ့ကြိုသင်တန်းအား ၂၀၀၁ မှ ၂၀၀၅ ခုနှစ်ထိ တတ်ရောက်ခဲ့ပါသည်။ ထိုဘွဲ့ကြိုသင်တန်းကာလအတွင်း ပြုလုပ်ခဲ့သည့် "Literature Review on Face Recognition Methods" ဟုခေါ်ဆိုသည့် စာတမ်းငယ် မှာ စာရေးသူ၏ ပထမဦးဆုံး မှတ်ဥာဏ်အတု Artificial Intelligence (AI) နှင့် ပတ်သတ်သည့် သုတေသန လုပ်ငန်းစဥ် တစ်ခု ဖြစ်ခဲ့ပါသည်။ ထိုသုတေသနမှ အစပြု၍ computer vision နှင့် machine learning ကို ပိုမို စိတ်၀င်စားလာကာ Computer Vision ဘာသာရပ်ကို အထူးပြု၍ မာစတာနှင့် ဒေါက်တာ ဘွဲ့များကို ဆက်လက် ရယူခဲ့ပါသည်။ သုတေသန စာတမ်းငယ်များစွာနှင့် Contextual Analysis of Videos နည်းပညာစာအုပ်ကို အင်္ဂလိပ်ဘာသာဖြင့်  ရေးသား ထုတ်၀ေခဲ့သည်။

၂၀၁၃ ခုနှစ် နှောင်းပိုင်းတွင် မြန်မာနိုင်ငံသို့ ပြန်ရောက်ပြီးနောက် မြန်မာလူငယ်များကို   computer vision နှင့် machine learning ဆိုင်ရာ ဘာသာရပ်များကို သင်ကြားဖြစ်ရင်းမှ အင်္ဂလိပ်ဘာသာစကားဖြင့် ထုတ်၀ေသည့် စာအုပ်များကို မြန်မာ လူငယ်များ (အထူးသဖြင့် အင်္ဂလိပ်စာ ဘာသာစကား အားနည်းသော လူငယ်များ ) အတွက် ဖတ်ရှု့လေ့လာရန် ခက်ခဲလျက်ရှိသည်ကို သိရှိလာရသည်။ သို့ဖြစ်ရာ ရိုးရှင်းသည့် စကားလုံးများကို အသုံးပြု၍ မြန်မာသာသာဖြင့် နည်းပညာစာအုပ်များကို သင်ကြားရေးတွင် အသုံးပြုရန် စတင် ရေးသားခဲ့ပါသည်။ ထို့နောက် ၂၀၁၈ ခုနှစ်တွင် Introduction to MATLAB: Learning by Doing စာအုပ်ကို လည်းကောင်း၊ ၂၀၂၃ ခုနှစ်တွင်  Introduction to Supervised Machine Learning စာအုပ်ကို လည်းကောင်း ရေးသားထုတ်၀ေခဲ့သည်။ 

ယခု ရေးသားထုတ်၀ေမည့် Introduction to Deep Learning နှင့်  ၂၀၂၃ ခုနှစ်တွင် ထုတ်၀ေခဲ့သည့် Introduction to Supervised Machine Learning - စာအုပ် ၂ အုပ်လုံးမှာ ကျွန်မ ဘ၀၏ စိတ်သောက အများဆုံး အချိန်တွင် ရေးသားခဲ့သည့် စာအုပ်များဖြစ်သည်ဟု ဆိုနိုင်ပါသည်။ လက်ရှိ မြန်မာနိုင်ငံ၏ အခြေအနေများမှာ ပညာရေးဖြင့် တိုင်းပြည်ကို တိုးတက်စေချင်သူ ကျွန်မအတွက် စိတ်ဖိစီးမှု များစွာ ဖြစ်စေပါသည်။ သို့သော် ခက်ခဲနေသည့် အခြေအနေများ အကြားတွင်ပင် မလျှော့သော ဇွဲ၊ လုံ့လဖြင့် ကြိုးစားနေကြသည့် မြန်မာ လူငယ်များမှ ပေးသောခွန်အားဖြင့် ဤနည်းပညာစာအုပ်များကို ရေးသားခဲ့ပါသည်။  ယခု စာအုပ်သည် အခြားနိုင်ငံသားများနည်းတူ မြန်မာ လူငယ်များလည်း နည်းပညာကို အမီလိုက်နိုင်စေရန် ရည်ရွယ်ရေးသားခြင်း ဖြစ်သည်။ နှလုံးသားနှင့် ဦးဏှောက် စည်းချက်ညီစွာဖြင့် ကြိုးစားနေကြသော မြန်မာ လူငယ်၊ လူရွယ် (ဆယ်လ်မွန်ငါး) များအတွက် အလွယ်တကူ ဖတ်ရှု့နိုင်သည့် နည်းပညာ စာအုပ်တစ်အုပ်အဖြစ် ပညာဖြန့်၀ေနိုင်ရန် ရည်ရွယ်ပါသည်။ 

ယခု စာအုပ်တွင် အပိုင်း ၄ ပိုင်းပါ၀င်ပြီး ပထမ အပိုင်းမှာ Deep Learning ကို မိတ်ဆက်ခြင်း ဖြစ်ပါသည်။ သို့သော် Machine Learning / AI စသည့် အခြေခံ သဘောတရားများကို ပြန်လည်ရှင်းပြခြင်းထက်  Deep Learning ၏ သဘောတရားများကို တိုက်ရိုက်မိတ်ဆက်သွားမည် ဖြစ်ရာ AI / Machine Learning စသည့် နည်းပညာ စကားလုံးများနှင့် ၀ေးကွာနေသူများအဖို့  ဤစာအုပ်ကိုမဖတ်မီ  Introduction to Supervised Machine Learning ကို ဖတ်ရှု့ရန် တိုက်တွန်းပါသည်။ Deep Learning ကို လေ့လာရာတွင် အရေးကြီးသည့် အချို့ သင်္ချာ ခေါင်းစဥ်များနှင့် ယခု စာအုပ်တွင် အသုံးပြုသွားမည့်  Python Library များကိုလည်း ပထမ အပိုင်းတွင် မိတ်ဆက်ထားပါသည်။ 

ဒုတိယအခန်းတွင်  Deep Learning ၏ မူလ အခြေခံဖြစ်သည့် Shallow Neural Network ကို ရှင်းပြသွားမည်ဖြစ်ပြီး လက်တွေ့ လေ့ကျင့်ခန်းများကို လည်း Python Code များနှင့်အတူ ဆွေးနွေးပေးသွားမည် ဖြစ်သည်။ ထို့နောက် တတိယ အခန်းတွင်မူ ပိုမို အဆင်မြင့်သည့် Deep Learning Architecture အမျိုးမျိုးကို ရှင်းလင်းသွားပြီး Convolutional Neural Networks (CNNs)၊ Recurrent Neural Networks (RNNs) များကို အသုံးပြု၍ Image များ  Text များကို အမျိုးအစား ခွဲခြားပုံများကို ဆွေးနွေးသွားပါမည်။ နောက်ဆုံးအပိုင်းတွင်မူ Deep Learning ၏ နောက်ဆုံး ပြောင်းလဲလျက်ရှိသည့် ခေါင်းစဥ်များနှင့် ဆက်လက်လေ့လာရမည်များကို ဆွေးနွေးသွားပါမည်။ 

ယခု စာအုပ်ကို အင်္ဂလိပ် - မြန်မာ ၂ ဘာသာဖြင့် ရေးသားထားပြီး စာမျက်နှာအလိုက် တွဲ၍ ဖတ်ရှု့သွားနိုင်မည် ဖြစ်သည်။ တိုက်ရိုက် ဘာသာပြန်ဆိုထားခြင်း မဟုတ်ဘဲ စာသားပြေပြစ်စေရန် အဓိက ရေးသားထားသည် ဖြစ်ရာ ဘာသာစကား တစ်ခုထဲကို အသုံးပြု ဖတ်ရှု့သူများ အနေဖြင့် အကျိုးရှိနိုင်သကဲ့သို့ ဘာသာစကား ၂ ခုလုံးကို တွဲ၍ ဖတ်ရှု့ပါကလည်း ပို၍ အကျိုးများနိုင်ပါသည်။ 

နည်းပညာစာအုပ်များကို ဖတ်ရှု့ရာတွင် အမှန်တကယ် နားလည်လိုသည်ဆိုပါက လက်တွေ့ လေ့ကျင့်ခန်းများနှင့် တွဲ၍ ဖတ်ရှု့ရန် လိုအပ်ပါသည်။ ဤစာအုပ်တွင် ပါရှိသည့် Python ဖြင့်ရေးသားထားသည့် လေ့ကျင့်ခန်းများကို အောက်ပါ \href{https://github.com/myothida/Intro-To-Supervised-Machine-Learning.git}{\textbf{GitHub}} လင့်တွင် ၀င်ရောက်၍ download ရယူပြီးလေ့ကျင့်စေလိုပါသည်။ 