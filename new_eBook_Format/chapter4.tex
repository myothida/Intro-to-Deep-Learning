\chapter{Recurrent Neural Networks (RNNs)}\label{sec:RNN}
This section introduces RNNs, a class of neural networks capable of modeling sequential data. It discusses the architecture of RNNs, including recurrent layers and different variants such as Long Short-Term Memory (LSTM) and Gated Recurrent Unit (GRU). Practical examples illustrate how RNNs can be used for tasks such as natural language processing, time-series forecasting, and sentiment analysis. 

\section{Long Short-Term Memory (LSTM) Networks}

\subsection{Case Study 1: Time Series Prediction}

\subsection{Case Study 2: Feedback sentiment analysis}

Feedback sentiment analysis is the process of using natural language processing (NLP) and machine learning techniques to determine the sentiment expressed in user feedback. The goal is to classify the feedback into categories such as positive, negative, or neutral, allowing businesses to gain insights into customer satisfaction and areas for improvement.

\begin{remark}
This work is inspired by a project assignment given to my Master students. In this course, students were tasked with developing a classification model to identify. While the codes and explanations are my original work, the project context was developed as part of the coursework to provide practical learning experiences.
\end{remark}

\section{Autoencoders}\label{sec:autoencoder}
Autoencoders are neural networks trained to reproduce their input data at the output layer. They consist of an encoder network that compresses the input data into a lower-dimensional representation (encoding) and a decoder network that reconstructs the original input from the encoding. Autoencoders are used for tasks such as dimensionality reduction, data denoising, and anomaly detection.

\section{Transformers}\label{sec:transformers}

Transformers are a type of deep learning architecture that has gained significant popularity in natural language processing (NLP) tasks. Introduced by Vaswani et al. in the paper "Attention is All You Need," transformers have revolutionized NLP by achieving state-of-the-art performance on a wide range of tasks.

The key innovation of transformers lies in their self-attention mechanism, which allows them to capture long-range dependencies in sequential data more effectively than traditional recurrent neural networks (RNNs) and convolutional neural networks (CNNs). Unlike RNNs, transformers process input data in parallel, making them highly efficient for training on large datasets.

\section{Generative Adversarial Networks (GANs)}\label{sec:GAN} 
\subsection{Generative Adversarial Networks (GANs)}

\subsection{Reinforcement Learning (RL)}

\subsection{Integration of GANs and RL} 

\newpage



\section{Further Reading}

\subsection{Courses and Lecture Notes}
\begin{enumerate}[1] 
    \item Prof. Philippe Rigollet, Mathematics Of Machine Learning, MIT OpenCourse \url{https://ocw.mit.edu/courses/18-657-mathematics-of-machine-learning-fall-2015/}
\end{enumerate}

\subsection{Books}
\begin{enumerate}[1] 
    \item Shalev-Shwartz, Shai, and Shai Ben-David. Understanding Machine Learning: From Theory to Algorithms. Cambridge University Press, 2014. ISBN: 9781107057135
\end{enumerate}

\subsection{Leading Journal and Conferences}

\begin{enumerate}[1] 
    \item The International Conference on Machine Learning (ICML)
    \item The Conference on Neural Information Processing Systems (NeurIPS)     
\end{enumerate}

